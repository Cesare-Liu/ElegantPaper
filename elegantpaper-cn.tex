%!TEX program = xelatex
% 完整编译: xelatex -> biber/bibtex -> xelatex -> xelatex
\documentclass[lang=cn,a4paper,newtx,bibtex]{elegantpaper}

\title{ElegantPaper: 一个优美的 \LaTeX{} 工作论文模板}
\author{作者1 \\ 某某大学/机构 \and 作者2 \\ 某某大学/机构}
\institute{\href{https://elegantlatex.org/}{Elegant\LaTeX{} 项目组}}

\version{0.11}
\date{\zhdate{2022/12/31}}

% 本文档命令
\usepackage{array}
\newcommand{\ccr}[1]{\makecell{{\color{#1}\rule{1cm}{1cm}}}}
\addbibresource[location=local]{HE.bib, bit-wise.bib} % 参考文献,不要删除

\begin{document}

\maketitle

\begin{abstract}
\keywords{Elegant\LaTeX{},工作论文,模板}
\end{abstract}




另一方面,CKKS的自举设计与BGV/BFV类型有所不同,其区别主要体现在模约化过程。
其自举过程将模约化看作一个分段的周期函数,然后使用另一个连续性的周期函数(如正弦函数)在多个离散区间对其近似。
由于CKKS方案仅支持加法和乘法的同态运算,所以问题转化为如何利用多项式对正弦函数进行逼近,已有的方案采用泰勒多项式\cite{HE351/CHKKS18}、切比雪夫多项式\cite{HE353}\cite{HE359/eurocrypt/LeeLLKN21}、最小二乘法\cite{HE364}、拉格朗日插值法\cite{HE363}等方法找到一个多项式函数用来逼近正弦函数。


FHEW\cite{HE410FHEW/DucasM21}
\printbibliography[heading=bibintoc, title=\ebibname]

\appendix
%\appendixpage
\addappheadtotoc

\end{document}
